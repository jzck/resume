\section{Projects}

\subsection{gago}
\noindent
\begin{minipage}{.15\textwidth}
\centerline{\includegraphics[width=20mm]{img/gago}}
\end{minipage}%
\hspace{5mm}
\begin{minipage}{.8\textwidth}
\justify
I am the happy maintainer of \texttt{gago}, a framework for implementing genetic algorithms with state-of-the-art techniques. The project is coded in Go and makes full use of it's concurrency features. The code is tested and documented, examples are provided. More information is available at \textcolor{green}{\url{http://gago.readthedocs.io/en/latest/}}.
\end{minipage}

\subsection{OpenBikes}
\noindent
\begin{minipage}{.15\textwidth}
\centerline{\includegraphics[width=20mm]{img/openbikes}}
\end{minipage}%
\hspace{5mm}
\begin{minipage}{.8\textwidth}
\justify
OpenBikes is an application for visualizing the number of bikes/spaces in bike stations in real time. The visualization is done with \texttt{LeafletJS} and a lot of JavaScript. The data is collected (\texttt{Python}), stored (\texttt{MongoDB}) and analyzed (\texttt{sklearn}) in order to make forecasts and advise users/cities. The project's source code can be found at is live at \textcolor{green}{\url{https://github.com/OpenBikes}}.
\end{minipage}

\subsection{TaxiSID}
\noindent
\begin{minipage}{.15\textwidth}
\centerline{\includegraphics[width=20mm]{img/taxisid}}
\end{minipage}%
\hspace{5mm}
\begin{minipage}{.8\textwidth}
\justify
Together with 70 other students we built a taxi booking software (mobile and web) during two weeks. I was head of the development section and took care of most of the programming decisions. We worked in a very lean way and applied AGILE principles to great success. The project was relayed in the local news; more information is available here: \textcolor{green}{\url{http://cmisid.github.io/2016/01/04/TaxiSID.html}}.
\end{minipage}
